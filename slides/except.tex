\section{Exceptions}


\subsection{Konzept}

\begin{frame}{Konzept}
	\begin{itemize}
		\item Ersetzen viele Fehlercodes bei Funktionen (nicht alle)
		\item In zeitkritischen Bereichen nur in Ausnahmefällen
		\item In nicht zeitkritischen Bereichen oefter nutzbar
		\item Setzen \enquote{sauberen} C++ code vorraus
		\item Inkorrekte Nutzung kann sehr viel Overhead verursachen
	\end{itemize}
\end{frame}

\subsection{Was ist eine Exception}

\begin{frame}[fragile]{Typ}

Jede Exception sollte eine eigene Klasse sein, die von std::exception (oder einer ihrer Unterklassen) erbt

	\begin{lstlisting}[]
#include <exception>

class ATinyKittenDied : std::runtime_error
{
public:
    ATinyKittenDied()
    :std::runtime_error("Armes Kaetzchen")
    {}
};
	\end{lstlisting}
\end{frame}

\subsection{Wie werfe ich eine Exception}

\begin{frame}[fragile]{So gehts (nicht richtig)}
	\begin{lstlisting}[]
int groesserNull = -32;

if(groesserNull <= 0) {
    throw ATinyKittenDied();
}
	\end{lstlisting}

Name der Exception hat nix mit dem Fehler zu tun, beheben wir das

\end{frame}
\begin{frame}[fragile]{So gehts richtig}
	\begin{lstlisting}[]
#include <exception>

class ZahlIstNichtGroesserNull : std::runtime_error
{
public:
    ZahlIstNichtGroesserNull()
    :std::runtime_error("Die Zahl muss groesser Null sein")
    {}
};

void testeZahl(int groesserNull)
{
    if(groesserNull <= 0) {
        throw ZahlIstNichtGroesserNull();
    }
    // ... do something
}
	\end{lstlisting}

\end{frame}

\begin{frame}[fragile]{Hier ist eine exception unpraktisch}
	\begin{lstlisting}[]
#include <exception>

class KeineNeueDaten : std::runtime_error
{
public:
    KeineNeueDaten()
    :std::runtime_error("Es liegen keine neue Daten an")
    {}
};

struct Daten{};

Daten getDataFromTCPSocket()
{
    if(datenMenge == 0) throw KeineNeueDaten();
    
    return neueDaten;
}
	\end{lstlisting}

\end{frame}

\begin{frame}[fragile]{Hier ist eine exception nützlich}
	\begin{lstlisting}[]
Socket openSocket(uint16_t port)
{
    if(portAlreadyOpen(port)) throw PortIsAlreadyUsed();
    if(thereIsNoSpoon()) throw YouAreInTheMatrix();
    
    return Socket(port);
}

ErrCode openSocket(Sock* sock, uint16_t port)
{
    if(!sock) return ERR_SOCK_IS_NULL;
    if(portAlreadyOpen(port)) return ERR_PORT_IN_USE;
    if(thereIsNoSpoon()) return ERR_MATRIX_HAS_YOU;
    sock->port = port;
    sock->stream = newStream(port);

    return SUCCESS;
}
	\end{lstlisting}

\end{frame}


\begin{frame}[fragile]{Konstruktor schlägt fehl}
	\begin{lstlisting}[]
class Socket
{
private:
	SomePortInfo * info;
public:
	Socket(uint16_t port) throw(PortIsAlreadyUsed, YouAreInTheMatrix)
	{
		if(portAlreadyOpen(port)) throw PortIsAlreadyUsed();
		if(thereIsNoSpoon()) throw YouAreInTheMatrix();
		info = new SomePortInfo(port);
	}
};

Socket * sock = nullptr;

try{
	sock = new Socket(99);
} catch(PortIsAlreadyUsed) {
	std::cout << "port already in use" << std::endl;
} catch(...) {
	std::cout << "an unkown error has ocurred" << std::endl;
}
	\end{lstlisting}

\end{frame}
