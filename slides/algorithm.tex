\section{STL-Algorithmen}


\subsection{Konzept}

\begin{frame}{Konzept}
	\begin{itemize}
		\item Auf viele Container anwendbar
		\item Gleiches interface
	\end{itemize}
\end{frame}

\subsection{Die Wichtigsten}

\begin{frame}{Die Wichtigsten}
	\begin{itemize}
		\item copy
		\item find
		\item fill
		\item for_each
		\item remove
		\item max
		\item max_element
		\item min
		\item min_element
		\item sort
		\item swap
	\end{itemize}
\end{frame}

\subsection{Nutzung}

\begin{frame}[fragile]{Nutzung}
	
	\begin{lstlisting}[escapechar=\$]
std::vector<int> liste = {1, 42, 83, 7, 12};
int c_liste[] = {1, 42, 83, 12, 7};
std::sort(liste.begin(), liste.end());
std::sort(c_liste, c_liste + sizeof(c_liste)/sizeof(int));
if(std::equal(liste.begin(), liste.end(), c_liste)) {
	std::cout << "liste == c_liste" << std::endl;
} else {
	std::cout << "liste != c_liste" << std::endl;
}
std::vector<int>::iterator it = std::find(liste.begin(), liste.end(), 3);

if(it == liste.end()) {
	std::cout << "zahl nicht gefunden" << std::endl;
} else {
	*it = 99;
}
	\end{lstlisting}
	
\end{frame}
