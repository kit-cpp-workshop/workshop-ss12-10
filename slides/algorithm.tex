\section{STL-Algorithmen}


\subsection{Konzept}

\begin{frame}{Konzept}
	\begin{itemize}
		\item Auf viele Container anwendbar
		\item Gleiches interface
	\end{itemize}
\end{frame}

\subsection{Die Wichtigsten}

\begin{frame}{Die Wichtigsten}
	\begin{itemize}
		\item copy
		\item find
		\item fill
		\item for\_each
		\item remove
		\item max
		\item max\_element
		\item min
		\item min\_element
		\item sort
		\item swap
	\end{itemize}
\end{frame}

\subsection{Nutzung}

\begin{frame}[fragile]{Nutzung}
	
	\begin{lstlisting}[basicstyle=\tiny]
std::vector<int> liste = {1, 42, 83, 7, 12};
int c_liste[] = {1, 42, 83, 12, 7};

std::sort(std::begin(liste), std::end(liste));
std::sort(std::begin(c_liste), std::end(c_liste));

if(std::equal(std::begin(liste), std::end(liste), std::begin(c_liste))) {
    std::cout << "liste == c_liste" << std::endl;
} else {
    std::cout << "liste != c_liste" << std::endl;
}

std::vector<int>::iterator it = std::find(std::begin(liste), std::end(liste), 3);

if(it == std::end(liste)) {
    std::cout << "zahl nicht gefunden" << std::endl;
} else {
    *it = 99;
}
	\end{lstlisting}
	
\end{frame}
